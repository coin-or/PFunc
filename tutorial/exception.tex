\section{Exception handling}
\label{sec:exception}

Checking for errors following function calls has traditionally been
under-rated. Many bugs in programs are due to improper handling of these errors
or the failure to detect the errors as soon as they occur. This problem is more
grave when it comes to parallel processing. PFunc provides a robust error
checking mechanism in both its C and C++ interface. In this section, we will
see examples of how this can be achieved in a user program.

\subsection{\Cpp{}}
PFunc continues the philosophy of C++ by providing robust exception handling
mechanisms. PFunc's exception handling mechanism delivers exceptions thrown by
a task across threads to the task that is waiting on it. In the case that the
task throwing the exception has more than one task waiting on it, the thrown
exception object is delivered to each of the waiting tasks.  All exceptions
thrown by PFunc are derived from the base type \code{pfunc::exception}. The
following example shows the use of PFunc's exception handling.

\begin{center}
\begin{minipage}{0.60\textwidth}
\begin{lstlisting}
struct my_fn_object { 
  void operator () { 
    try { 
      ... 
    }
    catch (const pfunc::exception& error) { 
      std::cout << "Description: " << error.what () << std::endl;
      std::cout << "Trace: " << error.trace () << std::endl; 
      std::cout << "Code: " << error.code () << std::endl;
    } 
  }
};
\end{lstlisting}
\end{minipage}
\end{center}

PFunc's exceptions extend \code{std::exception} to provide useful information
to the users. There are three primary methods that help users in determining
the cause of the error. These are:

\begin{center}
\tablefont
\begin{tabular}{|c|l|}
\hline
Method & Explanation \\
\hline
\func{pfunc::exception::what} & Describes the error in string format. \\
\hline
\func{pfunc::exception::trace} & Returns the stack trance of the calls 
                                   through which this exception object was 
                                   transported. \\
\hline
\func{pfunc::exception::code} & Useful when the exception was caused by
                                a system call failure and returns the error 
                                number. \\
\hline
\end{tabular}
\end{center}
                                
It is important to note that PFunc takes care to ensure that exceptions are
transported across thread boundaries so that the exception is delivered to the
calling function without loss of any information. This is an important as it
gives sequential semantics to the program. All the errors that PFunc throws are
of the type \code{pfunc::exception_generic_impl}, which is derived from
\code{pfunc::exception}. This class is implemented to ensure seamless transfer
of exceptions from one thread to another. Another important point to note is
that if, during execution, PFunc encounters any other exception (eg.,
\code{std::bad_alloc}), it converts it into an exception of the type
\code{pfunc::exception_generic_impl}.  This is done so as to enable transfer of
standard exceptions between threads. For performance, exception handling is
\textbf{disabled} by default and can be enabled using a compile time flag.

\subsubsection{Forwarding exceptions}
When an exception object is thrown by a task that is deeply nested, it is often
necessary to propogate this exception all the way to the top-level task. In order
to propogate exceptions up the stack, it is necessary to first convert them 
into objects of type \code{pfunc::exception}. Consider the following example
that demonstrates this use case:

\begin{center}
\begin{minipage}{0.60\textwidth}
\begin{lstlisting}
struct my_fn_object { 
  void operator () { 
    try { 
      ... 
    }
    catch (const pfunc::exception& error) { 
      pfunc::exception* clone = error.clone();
      clone->add_to_trace (": from my_fn_object at " PFUNC_FILE_AND_LINE()); 
      clone->rethrow ();
    } 
  }
};
\end{lstlisting}
\end{minipage}
\end{center}

In the above example, any error that is caught by \code{my_fn_object} is 
rethrown for higher-level tasks to catch. This is achieved using the following
functions:

\begin{center}
\tablefont
\begin{tabular}{|c|l|}
\hline
Method & Explanation \\
\hline
\func{pfunc::exception::clone} & Creates a replica of the exception object. \\
\hline
\func{pfunc::exception::add_to_trace} & Appends a string that is used by \func{trace}. \\
\hline
\func{pfunc::exception::rethrow} & Throw's the cloned exception object.\\
\hline
\end{tabular}
\end{center}

\subsection{C}
In C, there is no support for exceptions. All PFunc C APIs return an integer
that tells us how the function call proceeded. The error code returned by C
APIs are equivalent to that returned by \func{pfunc::exception::code} in
\Cpp{}.  PFunc defines a number of error values that \textbf{should} be checked
for to ensure that the calls to PFunc have succeeded. As PFunc does not store
the return value of the preceeding calls, it is unable to detect presence of
earlier errors. For more information, refer to the function documentation to
check the possible errors that can be returned by each call.  Here is an
example of how one might check for errors in C.

\begin{center}
\begin{minipage}{0.60\textwidth}
\begin{lstlisting}
void my_fn (void*) {
  if (PFUNC_SUCCESS != (error = pfunc_init (...))) {
    switch (error) {
      case PFUNC_INITIALIZED: /* error message */
                              break;
      case PFUNC_NOMEM: /* error message */
                        break; 
      case PFUNC_ERROR: /* error message */
                        break;
      default: break;                  
    }
  }
}
\end{lstlisting}
\end{minipage}
\end{center}

Note that all the error values returned by the C interface are less than zero
and \textbf{do not} clash with the system error codes such as EINVAL, EBUSY,
etc.  
