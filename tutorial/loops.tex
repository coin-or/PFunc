\section{Loop Constructs}
\label{sec:loops}

Loop parallelism is an important form of parallelism that often results in
dramatic speedups. 
%
In fact, constructs such as OpenMP's "parallel for" have been exclusively
dedicated to parallelizing for loops, which occur frequently in HPC
applications.
%
Task parallelism is a powerful form of parallelism that subsumes loop
parallelism. 
%
In this section, we discuss four experimental constructs provided in PFunc 
to simplify parallelizing various loop constructs.
%
As they are experimentatl, these features are made available under the special
header files: \code{pfunc/experimental/parallel_for.hpp>},
\code{pfunc/experimental/parallel_reduce.hpp>}, and
\code{pfunc/experimental/parallel_while.hpp>}.

\subsection{For Loops}
\label{subsec:for}
Consider a standard \code{for} statement that iterates over a \textit{randomly
accessible} set of elements. 
%
It is quite important that the elements be randomly accessible because
parallelization may fail to yield significant performance boost if iteration
(that is, ``advancing the pointer'') takes longer than the computation itself.
%
We can devise an elegant divide and conquer mechanism to parallelize the 
computations in the following manner:

\begin{itemize}
\item At each level (starting with level 0), inspect the iteration space to
determine benefit of parallelization.
\item If parallelization will help, split the interval into two and execute
iterations over the split iteration space in parallel.
\item Repeat until the number of iterations in the iteration space are too few
to benefit from parallelization --- execute this space serially.
\end{itemize}

\paragraph{Space} In the true spirit of generic design, we devise the concept
of \code{Space} to as follows:

\begin{center}
\begin{minipage}{0.7\textwidth}
\begin{lstlisting}
concept Space<typename Model> : CopyAssignable <Model> {
  typename subspace_container;
  requires Sequence<subspace_container>;
@\halfline@ 
  const static size_t arity;
  const static size_t dimension;
@\halfline@ 
  size_t Model::begin() const;
  size_t Model::end() const;
  bool Model::can_split() const;
  subspace_container split() const;
}
\end{lstlisting}
\end{minipage}
\end{center}

In other words, a ``space'' must define how many ways it can be split
(\code{arity}) and its dimension (eg., 1-D, 2-D, etc).
%
Furthermore, it must define \func{begin} and \func{end}, which provide 
iterators to the beginning and end of the iteration space.
%
To help parallelization, every model of space must provide \func{can_split}
and \func{split} that help split the iteration space into \code{arity} chunks.
%
For example, consider a 1-D space, which provides a 2-way split and has a 
base case of 25 elements (i.e., if there are fewer than 25 elements in the 
iteration space, they are executed serially).
%
If such a space is initialized with the half-open interval $[0, 100)$, the 
following execution sequence occurs.

\begin{center}
\begin{minipage}{0.6\textwidth}
\begin{itemize}
\item[] $[0,100)$ --- split.
\item[] $[0,50)$, $[50,100)$ --- split.
\item[] $[0,25)$, $[25,50)$, $[50,75)$, $[75,100)$ --- no split.
\end{itemize}
\end{minipage}
\end{center}

\paragraph{\code{pfunc::parallel_for}} This is a function object akin to 
\func{for_each} in STL.
%
\code{pfunc::parallel_for} Takes in a space and a functor, and executes the 
functor over the given space in parallel.
%
The assumption is that the functor has the access to the entire container
and hence all the harness needs to do is provide access to the correct 
iteration range.
%
The functor must be a model of the \code{ForExecutable} concept given below:

\begin{center}
\begin{minipage}{0.6\textwidth}
\begin{lstlisting}
concept ForExecutable<typename Model, typename SpaceType> : 
  Space<SpaceType>, CopyAssignable <Model> {
 void Model::operator()(const SpaceType&) const;
}
\end{lstlisting}
\end{minipage}
\end{center}

\paragraph{Example} Consider the task of scaling each element of a 
\code{std::vector} by a constant factor.
%
The functor that is needed to execute this scaling operation is given below:
%
\begin{center}
\begin{minipage}{0.8\textwidth}
\begin{lstlisting}
struct vector_scale {
  private:
  std::vector<double>& my_vector;
  double scaling_factor;
@\halfline@
  public:
  vector_scale (std::vector<double>& my_vector, const double scaling_factor) :
    my_vector (my_vector), scaling_factor (scaling_factor) {}
@\halfline@
  void operator() (const pfunc::space_1D& space) const {
    for (size_t i = space.begin(); i<space.end(); ++i) {
      my_vector[i] *= scaling_factor;
    }
  }
};
\end{lstlisting}
\end{minipage}
\end{center}
%
Notice that \code{vector_scale} is a model of \code{ForExecutable} concept,
and can be used with \code{pfunc::parallel_for}.
%
As the iteration space defined by \code{std::vector} is 1-D, we use PFunc's 
built-in \code{space_1D}, which is a model of \code{Space} concept.
%
Next, we define the PFunc instance to be used in parallelization of the
\code{for} loop:

\begin{center}
\begin{minipage}{0.8\textwidth}
\begin{lstlisting}
typedef 
pfunc::generator <pfunc::cilkS, /* Cilk-style scheduling */
                  pfunc::use_default, /* No task priorities needed */
                  pfunc::use_default /* any function type*/> generator_type;
typedef generator_type::attribute attribute;
typedef generator_type::task task;
typedef generator_type::taskmgr taskmgr;
\end{lstlisting}
\end{minipage}
\end{center}
%
Notice that we have to use \code{pfunc::use_default} as the type of the functor
because of the way in which \code{pfunc::parallel_for} is defined.
%
Finally, we invoke \code{pfunc::parallel_for} on the entire iteration space in
the following manner:
\begin{center}
\begin{minipage}{0.7\textwidth}
\begin{lstlisting}
taskmgr global_taskmgr (/*nqueues*/, /*nthreads-per-queue*/);
task for_loop_task;
attribute for_loop_attribute (false /*nested*/, false /*grouped*/);
pfunc::parallel_for<generator_type, vector_scale, pfunc::space_1D> 
  for_loop (pfunc::space_1D (0,n), 
            vector_scale (my_vector, scaling_factor), 
            global_taskmgr);
pfunc::spawn (global_taskmgr, for_loop_task, for_loop_attribute, for_loop);
pfunc::wait (global_taskmgr, for_loop_task);
\end{lstlisting}
\end{minipage}
\end{center}
%
For the complete example, please see \code{examples/for.cpp}.

\paragraph{\code{pfunc::parallel_reduce}} An important variation of a simple 
\code{for} loop is the ability to \textit{reduce} the values in a collection
to a single value.
%
Examples for such operation include finding the sum of element in a vector, 
finding the minimum element in a vector, etc.
%
To execute such loops in parallel, PFunc provides \code{pfunc::parallel_reduce} 
construct, which is a slight variation of \code{pfunc::parallel_for}.
%
Firstly, the functor that executes the reduction operation is required to be
a model of \code{ReduceExecutable} concept that is defined below:

\begin{center}
\begin{minipage}{0.8\textwidth}
\begin{lstlisting}
concept ReduceExecutable<typename Model, typename SpaceType> : 
             Space<SpaceType>, Assignable<Model>, CopyAssignable <Model> {
  Model Model::split () const;
  void Model::join (const Model&);
  void Model::operator () (const SpaceType&);
}
\end{lstlisting}
\end{minipage}
\end{center}
%
Notice that \code{ReduceExecutable} requires \func{split} and \func{join} 
functions in addition to \func{operator()}.
%
Furthermore, \func{operator()} is non-\code{const} to allow it to modify 
internal state of the functor.
%
To better understand, let us consider the example of computing the sum of 
elements in a \code{std::vector}; the code is given below.
%
\begin{center}
\begin{minipage}{0.7\textwidth}
\begin{lstlisting}
struct accumulate {
  private:
  std::vector<double>& my_vector;
  double sum;
@\halfline@
  public:
  accumulate (std::vector<double>& my_vector, const double init) :
    my_vector (my_vector), sum (init) {}
  void operator() (const pfunc::space_1D& space) {
    for (size_t i = space.begin(); i<space.end(); ++i) sum += my_vector[i];
  }
  accumulate split () const { return accumulate (my_vector, 0.0); }
  void join (const accumulate& other) { sum += other.get_sum (); }
  double get_sum () const { return sum; }
};
\end{lstlisting}
\end{minipage}
\end{center}
%
As expected, \func{operator()} simply accumulates the sum of elements in its 
iteration space.
%
The crucial portion functions required for parallelization are \func{split} 
and \func{join}.
%
When \code{pfunc::parallel_reduce} determines that the iteration space needs 
to be split because there is an oppotunity for parallelism, it calls
\func{split} on the functor; \func{split} is expected to create a properly 
initialized copy of the functor.
%
In the case of \code{accumulate}, proper initialization is done by setting the 
\code{sum} to \code{0.0}.
%
Similarly, when all the parallel chunks have completed iteration, the partial 
results are added up using the \func{join} operation.
%
In the case of \code{accumulate}, \func{join} merely adds up the partial 
sums. 
%
Finally, we invoke \code{pfunc::parallel_reduce} on the entire iteration space
in the following manner:

\begin{center}
\begin{minipage}{0.7\textwidth}
\begin{lstlisting}
taskmgr global_taskmgr (nqueues, threads_per_queue_array);
@\halfline@
task accumulate_task;
attribute accumulate_attribute (false /*nested*/, false /*grouped*/);
accumulate root_accumulate (my_vector, 0.0);
pfunc::parallel_reduce<generator_type, accumulate, pfunc::space_1D> 
  root_reduce (pfunc::space_1D (0,n), root_accumulate, global_taskmgr);
@\halfline@
pfunc::spawn (global_taskmgr, accumulate_task, accumulate_attribute, root_reduce);
pfunc::wait (global_taskmgr, accumulate_task);
\end{lstlisting}
\end{minipage}
\end{center}
%
For the complete example, please see \code{examples/reduce.cpp}.

\subsection{While Loop}
\label{subsec:while}
%
\code{pfunc::parallel_for} operates when the collection of elements over which
we iterate allows random access. 
%
That is, if A is the collection of elements, then, we can access A[i] in
constant time. 
%
This property does not hold true for many data structures such as linked lists
and trees. 
%
If the computation involved when processing every element in such data
structures is sufficiently large, then it is benefical to parallelize the
execution of such loops.
%
Due to the nature of the data structures involved, we do not think of 
parallelizing over the iteration space, but rather in terms of processing 
each element in parallel. 
%
Since we do not know the number of elements that need to be processed, we the 
parallelization resembles a \code{while} loop; hence, the control structure 
is called \code{pfunc::parallel_while}.
%
The operation performed by \code{pfunc::parallel_while} is equivalent to the 
operation performed by \func{serial_while} given below:;
%
\begin{center}
\begin{minipage}{0.7\textwidth}
\begin{lstlisting}
template <typename InputIterator, typename Functor>
void serial_while (InputIterator first, InputIterator last, Functor func) {
  while (first != last) func (*first++);
}
\end{lstlisting}
\end{minipage}
\end{center}
%
Similar to \func{serial_while}, \code{parallel_while} takes as input, a pair
of iterators (\code{first} and \code{last}) and a functor in addition to the
PFunc library instance that is used for parallelization.
%
\code{pfunc::parallel_while} iterates through the collection contained in
(\code{first}, \code{last}] and spawns a task to process each element in 
the collection.
%
Of course, this scheme assumes that processing each task is independent of one
another.
%
For a functor to be parallelized using \code{pfunc::parallel_while}, it has to 
model \code{WhileExecutable} given below:
%
\begin{center}
\begin{minipage}{0.7\textwidth}
\begin{lstlisting}
concept WhileExecutable <typename Model, typename ArgumentType> : 
                                           CopyAssignable <Model> {
  typename argument_type;
  is_convertible <argument_type, ArgumentType>;
  void Model::operator() (ArgumentType) const;
}
\end{lstlisting}
\end{minipage}
\end{center}
%
To better understand parallelizing while loops, consider a simple example of
parallelizing a loop that applies \func{big_func} to each element of a
\code{std::list}.
%
From our discussion above, \func{big_func} must be a model of
\code{WhileExecutable} and is defined as follows:
%
\begin{center}
\begin{minipage}{0.45\textwidth}
\begin{lstlisting}
template <typename T>
struct big_func { 
  typedef T* argument_type;
  void operator() (T* arg) const { ... }
};
\end{lstlisting}
\end{minipage}
\end{center}
%
Once we have defined \func{big_func}, we can parallelize application of 
\code{big_func} to the elements of an \code{std::list} as follows:
%
\begin{center}
\begin{minipage}{0.7\textwidth}
\begin{lstlisting}
struct object {...};
std::list <object*> my_list;
// populate my_list
@\halfline@
taskmgr global_taskmgr (nqueues, threads_per_queue_array);
task while_task;
attribute while_attribute (false /*nested*/, false /*grouped*/);
pfunc::parallel_while<generator_type, std::list<object*>::iterator, foo<object> > 
  root_while (my_list.begin(), my_list.end(), big_func<object>(), global_taskmgr);
@\halfline@
pfunc::spawn (global_taskmgr, while_task, while_attribute, root_while);
pfunc::wait (global_taskmgr, accumulate_task);
\end{lstlisting}
\end{minipage}
\end{center}
