\section{Choosing The Right PFunc}
\label{sec:generate}
%
\begin{table}
\begin{center}
\begin{tabular}{|c|c|}
\hline
Feature & Default \\
\hline
\code{Scheduling policy} & \code{cilkS} \\
\hline
\code{Compare} & \func{std::less<int>} \\
\hline 
\code{Function object} & \code{struct \{ virtual void operator()() = 0; \};} \\
\hline
\end{tabular}
\end{center}
\caption{Default values for PFunc's template parameters.}
\label{tbl:default}
\end{table}
%
PFunc is a templated library; the first step is, therefore, to generate the 
concrete type that will be used as the library instance (\Cpp{} only).
%
PFunc takes three template parameters: \code{scheduling policy},
\code{compare}, and \code{function object}.
%
For most users, it is sufficient to provide default values to the template 
parameters that are used in PFunc; Table~\ref{tbl:default} lists the default
values for each of the three template parameters.
%
We briefly describe the roles of each of these template parameters; for a more
detailed description, please see Sections~\ref{sec:design} and~\ref{sec:custom}.
%
\begin{list}{\labelitemi}{\leftmargin=0em}
% Scheduling policy
\item \code{Scheduling policy:}
This template parameter names the scheduling policy to be used; the built-in
values that can be used are \code{cilkS}, \code{lifoS}, \code{fifoS}, and
\code{prioS}.

% Compare
\item \code{Compare:}
This template parameter represents the ordering operator for task priorities; 
for the built-in scheduling policies, it is used only for \code{prioS}.

% Functor
\item \code{Function object:}
This template parameter determines the type of the function objects that are 
parallelized; when default value is choosen for this parameter, all function 
objects the need to be parallelized are required to inherit from a abstract
base class.
\end{list}
%
The code below summarizes how a library instance description can be 
generated.
%
\begin{center}
\begin{minipage}{0.75\textwidth}
\begin{lstlisting}
typedef pfunc::generator<cilkS, /* scheduling policy */
                         pfunc::use_default, /* compare */
                         pfunc::use_default> my_pfunc; /*function object*/
\end{lstlisting}
\end{minipage}
\end{center}
%
Here, we have generated an new library instance description of PFunc by
choosing the Cilk-style scheduling policy. 
%
The values for the \code{compare} and \code{function object} features are
allowed to be defaults; In fact, It is possible to use
\code{pfunc::use_default} for all the features (Figure~\ref{tbl:default}).  
%
PFunc automatically chooses sensible values for the features in this case.  
%
The type \code{my\_pfunc} thus generated in our example
is a custom instance that can be used to parallelize user applications. 
%
In PFunc, there are four important types that users are exposed to:
\code{attribute}, \code{group}, \code{task} and \code{taskmgr}.  
%
Once the required library instance description has been generated, these types
can be accessed as follows:

\begin{center}
\begin{minipage}{0.4\textwidth}
\begin{lstlisting}
typedef my_pfunc::attribute attribute; 
typedef my_pfunc::group group; 
typedef my_pfunc::task task; 
typedef my_pfunc::taskmgr taskmgr; 
\end{lstlisting}
\end{minipage}
\end{center}

% attribute
Objects of type \code{attribute} allow users to control the execution of
spawned tasks by setting attributes such as task priority and task affinity
(see Section~\ref{sec:attribute}).
% group
Objects of type \code{group} can be used to create collaborations of tasks that
can communicate with each other using point-to-point message passing and
barrier synchronization (see Section~\ref{sec:group}).
% task
Objects of type \code{task} are used as references to spawned tasks, which 
can be passed to other tasks. The ability to pass task references is crucial
for the support of multiple task completion notifications (see
Section~\ref{sec:spawn}).
% taskmgr
Finally, objects of type \code{taskmgr} manage threads and their task
queues, and are responsible for task scheduling (see Section~\ref{sec:spawn}).

\subsection{C}
In C, both because of the lack of support for generic programming and the
pitfalls of over-using preprocessor macros, PFunc pre-generates the library
instance descriptions for the users; the definitions for these are present in
\code{pfunc/pfunc.h}.
%
The users are merely required to then select the right set of functions from
the available sets of pre-generated functions; each set is denoted by a 
common prefix and is given in the table below:
%
\begin{center}
\begin{tabular}{|c|c|c|c|}
\hline
Instance description & Scheduling policy & Compare, priority & Function type \\
\hline
\code{pfunc_cilk_*} & Cilk-style & \code{unused} & \code{void (*)(void*)} \\
\hline
\code{pfunc_lifo_*} & Queue & \code{unused} & \code{void (*)(void*)} \\
\hline
\code{pfunc_fifo_*} & Stack & \code{unused} & \code{void (*)(void*)} \\
\hline
\code{pfunc_prio_*} & Priority-based & \code{<} op, \code{int} & \code{void (*)(void*)} \\
\hline
\end{tabular}
\end{center}
%
Like in \Cpp{}, there are four important types exposed to the users:
\code{attribute}, \code{group}, \code{task} and \code{taskmgr}.
%
For example, for the Cilk-style library instance description, the names by
which these types can be accessed are \code{pfunc_cilk_attr_t},
\code{pfunc_cilk_group_t}, \code{pfunc_cilk_task_t} and
\code{pfunc_cilk_taskmgr_t}. 

\paragraph{Caveat} 
As PFunc is implemented completely in \Cpp{}, the C types (\code{attribute,
group, task, taskmgr}) are mere typed pointers to their \Cpp{} counterparts. 
%
Consequently, these types (eg., \code{pfunc_cilk_attr_t},
\code{pfunc_cilk_group_t}, \code{pfunc_cilk_task_t} and
\code{pfunc_cilk_taskmgr_t}) need to be initialized and cleared explicility
using calls to their respective \func{init} and \func{clear} functions. 
%
This notion of initializing and clearing PFunc's C types will be reinforced
throughout the C examples described in this tutorial.
