\section{Synchronization Primitives}
\label{sec:sync}
%
In PFunc, we encourage parallel programming without using low-level constructs
such as locks and atomic operations as these constructs often interfere with
task scheduling.
%
However, discerning users can make use of these constructs to improve
performance of their applications.
%
For this purpose, PFunc provides portable locks and low-level atomic
instructions; however, we do not provide access to condition variables as they
interfere exceedingly with task scheduling; in this section, we briefly explain
these constructs.
%
Since synchronization primitives are a secondary goal in PFunc, all the
relevant functions are prototyped in \code{pfunc/pfunc_atomics.h}, which should
be included to use these functions.

\subsection{pfunc::mutex}
\label{subsec:mutex}
%
\code{pfunc::mutex} is a \Cpp{} class that implements a portable lock that
provides \func{lock}, \func{unlock}, and \func{trylock} operations on all
supported platforms.
%
All locking operations occur at thread-scope; that is, when a task calls
\func{lock}, it blocks the thread executing the task till the lock can be 
acquired.
%
Due to this reason, users are encouraged to use \func{trylock}, a non-blocking
function instead of \func{lock}.
%
Note that \lstinline{foo.lock()}, where \code{foo} is a \code{pfunc::mutex}, is
theoretically the same as \lstinline{while (false==foo.trylock());}; however, 
\func{lock} can save computational cycles by putting the calling thread to 
sleep whereas repeated calls to \func{trylock} spin the CPU.
%
The precise implementation of \code{pfunc::mutex} depends on the platform; when
\textit{futexes}, a type of user-level fast locks, are supported ($\ge{}$ Linux
kernel 2.6), \code{pfunc::mutex} is designed to use them.
%
In all other cases, \code{pfunc::mutex} uses either \textit{pthread} mutexes or
in the case of Windows, \textit{native} locks.
%
Like most other features in PFunc, users can choose to implement their own 
mutexes and use that instead of \code{pfunc::mutex}.
%

\subsection{Atomic operations}
\label{subsec:atomic}
%
An alternative to using lock-based algorithms is to make use of lock-free
algorithms; these algorithms make use of \textit{atomic operations} such as
\textit{compare-and-swap} to ensure atomicity of updates instead of resorting
to locks.
%
PFunc provides four portable atomic operations on 8, 16, 32, and 64 bits.
%
\paragraph{Compare-and-swap} This is a key operation in many lock-free
algorithms, including PFunc's futex-based implementation of
\code{pfunc::mutex}. 
%
The operation performed by compare-and-swap is given in pseudo-code below.
 
\begin{center}
\begin{minipage}{0.7\textwidth}
\begin{lstlisting}
intX_t pfunc_compare_and_swap_X (volatile void* dest, /*mem location*/
                                 intX_t exchg, /*new value*/
                                 intX_t comprnd) { /*old value*/
  if (*dest == comprnd) { *dest = exchg; return exchg; } 
  else return *dest;
}
\end{lstlisting}
\end{minipage}
\end{center}

In the above example, \code{X} denotes the number of bits to compare-and-swap;
the valid values are 8, 16, 32, and 64.

\paragraph{Fetch-and-add} This primitive allows programmers to atomically 
read and update 8, 16, 32, or 64 bit values, and hence, is an important 
operation to support.
%
The operation performed by fetch-and-add is given in pseudo-code below.

\begin{center}
\begin{minipage}{0.7\textwidth}
\begin{lstlisting}
intX_t pfunc_fetch_and_add_X (volatile void* location, /*mem location*/
                              intX_t addend) { /*to add*/
  result = *location; *location += addend; return result;
}
\end{lstlisting}
\end{minipage}
\end{center}

\paragraph{Fetch-and-store} This primitive allows programmers to atomically 
read and replace 8, 16, 32, or 64 bit values, which is a slight modification 
to fetch-and-add.
%
The operation performed by fetch-and-store is given in pseudo-code below.

\begin{center}
\begin{minipage}{0.7\textwidth}
\begin{lstlisting}
intX_t pfunc_fetch_and_store_X (volatile void* location, /*mem location*/
                                intX_t new_val) { /*to store*/
  result = *location; *location += new_val; return result;
}
\end{lstlisting}
\end{minipage}
\end{center}

\paragraph{Read-with-fence} This primitive allows programmers to read the most
current 8, 16, 32, or 64 bit value at a memory location by inserting a memory
fence just before the read operation. 
%
The PFunc nomenclature for this function is \func{pfunc_read_with_fence_X}, 
where \code{X} is 8, 16, 32, or 64.

\paragraph{Write-with-fence} This primitive allows programmers to write a 8,
16, 32, or 64 bit value to a memory location and then ensure that the value is
flushed down to memory (i.e., not just written to the cached value) by placing
a memory fence right after the write operation. 
%
The PFunc nomenclature for this function is \func{pfunc_write_with_fence_X}, 
where \code{X} is 8, 16, 32, or 64.
