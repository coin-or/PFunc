\section{Performance profiling}
\label{sec:perf}

PFunc is fully integrated with the Performance Application Programming
Interface (PAPI), thereby allowing users to profile their
applications with ease. PAPI was chosen due to its wide availability and
portability across many hardware platforms. Profiling is handled mainly through
the \code{taskmgr} type. Users can specify the events (both PAPI presets and
native) that they wish to monitor when initializing objects of the
\code{taskmgr} type. Consider the sample code given below:

\begin{lstlisting}
typedef pfunc::generator<cilkS, /* scheduling policy */
                         pfunc::use_default, /* compare */
                         pfunc::use_default> my_pfunc; /*function object*/

enum {nthds=2,
      nevents=2
};

struct my_perf_data: pfunc::perf_data {
  perf_data::event_value_type storage[nthds][nevents];
  int events[nevents];

  my_perf_data () {
    events[0] = PAPI_L1_DCA; 
    events[1]= PAPI_L1_DCM;
  }

  int get_num_events () const { return nevents; }

  int* get_events () const { return events; }

  perf_data::event_value_type** get_event_storage () const {
    return storage; 
  }
};

my_perf_data perf;
my_pfunc::taskmgr cilk_tmanager (/*nqueues*/, /*nthreads_per_queue*/, perf);
\end{lstlisting}

In this example, we utilize PFunc to measure L1 data cache behavior. We
derive from \code{perf\_data} to communicate the required
measurements to PFunc. PFunc stores the requested event values in
\code{my\_perf\_data} and these values can subsequently be used for performance
tuning.

\paragraph{Caveat} To enable performance profiling, please ensure that PAPI 
is installed on your system and that \code{USE_PAPI} flag is turned 
\code{ON|On} during configuration using CMake.
