\documentclass{beamer}
\usepackage{graphicx}
\usepackage{listings}

\lstdefinestyle{basic}{showstringspaces=false,
                       columns=fullflexible,
                       language=C++,
                       escapechar=@,
                       basicstyle=\tiny\sffamily,
classoffset=1,
morekeywords={pfunc_cilk_task_init, pfunc_cilk_task_clear},
morekeywords={pfunc_cilk_spawn_c, pfunc_cilk_wait, pfunc_cilk_task_t},
keywordstyle=[1]\color{blue}, 
classoffset=0,
moredelim=**[is][\color{white}]{~}{~},
literate={->}{{$\rightarrow\;$}}1 {<-}{{$\leftarrow\;$}}1 {=>}{{$\Rightarrow\;$}}1,
}
\lstset{language=C++,style=basic}

\newenvironment{discuss}{\begin{list}{}{}\item[]{\it Discussion item:}
\addcontentsline{toc}{subsection}{Discussion Item}
}{{\rm ({\it End of discussion item.})} \end{list}}
\newcommand{\code}[1]{\lstinline[basicstyle=\sffamily]{#1}}
\newcommand{\func}[1]{\lstinline[basicstyle=\sffamily]{#1()}}
\newcommand{\Cpp}{C\kern-0.05em\texttt{+\kern-0.03em+}}
\newcommand{\MPIpp}{MPI\kern-0.05em\texttt{+\kern-0.03em+}}
\newcommand{\tablefont}{\fontsize{8}{13}\selectfont}
\newcommand{\halfline}{\vspace{-1.0ex}}

\mode<presentation>
{
  \usetheme{Warsaw}
  \setbeamercovered{transparent}
  \definecolor{TUCGreen}{RGB}{0,100,50}
  \definecolor{TUCGreenlight}{RGB}{120,200,150}
  \definecolor{IURed}{RGB}{255,0,0}
  \definecolor{IURedlight}{RGB}{255,204,204}
  \definecolor{OMPIBlue}{RGB}{80,130,255}
  \definecolor{OMPIBluelight}{RGB}{240,240,255}
  \definecolor{white}{RGB}{255,255,255}
  \definecolor{black}{RGB}{0,0,0}
  \setbeamercolor*{palette primary}{fg=white, bg=IURed}
  \setbeamercolor*{palette secondary}{fg=white, bg=IURed}
  \setbeamercolor*{palette tertiary}{fg=white, bg=IURed}
  \setbeamercolor*{palette quartenary}{fg=white, bg=IURed}
  \setbeamercolor*{palette sidebar primary}{fg=white, bg=IURed}
  \setbeamercolor*{palette sidebar secondary}{fg=white, bg=IURed}
  \setbeamercolor*{palette sidebar tertiary}{fg=white, bg=IURed}
  \setbeamercolor*{palette sidebar quartenary}{fg=white, bg=IURed}
  \setbeamercolor*{normal text}{fg=black, bg=white}
  \setbeamercolor*{example text}{fg=black, bg=white}
  \setbeamercolor*{titlelike}{fg=white, bg=IURed}
  \setbeamercolor*{separation line}{fg=white, bg=IURed}
  \setbeamercolor*{item projected}{fg=white, bg=IURed}
  \setbeamercolor*{block title}{fg=white, bg=IURed}
  \setbeamercolor*{block title alerted}{fg=IURed, bg=white}
  \setbeamercolor*{block title example}{fg=IURed, bg=white}
  \setbeamercolor{block body}{fg=black, bg=IURedlight}
  \setbeamercolor*{structure}{fg=IURed, bg=white}
  \setbeamercolor*{sidebar}{fg=IURed, bg=white}
  % or whatever (possibly just delete it)
}

\setbeamertemplate{background canvas}

\usepackage[english]{babel}
\usepackage[latin1]{inputenc}
\usepackage{times}
\usepackage[T1]{fontenc}

\title{The PFunc Implementation of NAS Parallel Benchmarks.}

\author{Shashi Kumar Nanjaiah}

\date{{\small Calfornia State University,}
{\footnotesize Sacramento, CA}}

\begin{document}

\begin{frame}
  \titlepage
\end{frame}

\begin{frame}
\frametitle{Overview.}
\begin{center}
\textcolor{blue}{The goal of this project is to prove the efficacy of task 
parallelism in PFunc to parallelize industry-standard benchmark computation 
kernels and applications on shared-memory.}
\end{center}
\begin{itemize}
\item Introduce PFunc, a new tool for task parallelism.
  \begin{itemize}
  \item New features and extensions.
  \item Fibonacci example.
  \end{itemize}
\item Introduce NAS parallel benchmarks.
  \begin{itemize}
  \item Briefly explain the 7 benchmarks.
  \end{itemize}
\end{itemize}
\end{frame}

\begin{frame}
\frametitle{PFunc: A new tool for task parallelism.}
\begin{itemize}
\item Extends existing task parallel feature-set. 
  \begin{itemize}
  \item Cilk, Threading Building Blocks, Fortran M, etc.
  \end{itemize}
\item Portable.
  \begin{itemize}
  \item Linux, OS X, AIX and Windows.
  \end{itemize}
\item Customizable.
  \begin{itemize}
  \item Generic and generic programming techniques.
    \begin{itemize}
    \item No runtime penalty.
    \end{itemize}
  \end{itemize}
\item C and \Cpp{} APIs.
\item Released under Eclipse Public License v1.0.
  \begin{itemize}
  \item \textcolor{blue}{http://coin-or.org/projects/PFunc.xml}
  \end{itemize}
\end{itemize}
\end{frame}

\begin{frame}[fragile]
\frametitle{Example: Parallelizing Fibonacci numbers.}
\begin{center}
\begin{minipage}{0.75\textwidth}
\begin{lstlisting}
typedef struct {int n; int fib_n;} fib_t;
@\halfline@
void fibonacci (void* arg) {
  fib_t* fib_arg = (fib_t*) arg;
@\halfline@
  if (0 == fib_arg->n || 1 == fib_arg->n) { 
    fib_arg->fib_n = fib_arg->n;
  } else { 
    pfunc_cilk_task_t fib_task;
    fib_t fib_n_1 = {(fib_arg->n)-1, 0};
    fib_t fib_n_2 = {(fib_arg->n)-2, 0};
@\halfline@
    pfunc_cilk_task_init (&fib_task);   
@\halfline@
    pfunc_cilk_spawn_c (fib_task, /* Handle to the task */
                         NULL,       /* Attribute -- use default */
                         NULL,       /* Group -- use default */
                         fibonacci,  /* Function to execute */ 
                         &fib_n_1);  /* Argument */
    fibonacci (&fib_n_2);
    pfunc_cilk_wait (fib_task);
@\halfline@
    pfunc_cilk_task_clear (&fib_task);   
@\halfline@
    fib_arg->fib_n = fib_n_1.fib_n + fib_n_2.fib_n;
  } 
}
\end{lstlisting}
\end{minipage}
\end{center}
\end{frame}

\begin{frame}[fragile]
\frametitle{Fibonacci: task creation overhead.}
$\rightarrow{}$ \textcolor{blue}{Fibonacci number 37 ($2^{36}\approx{}69$ billion tasks).}
\tablefont
\begin{center}
\begin{tabular}{|c|c|c|c|c|} 
\hline
Threads & Cilk Time (secs) & PFunc/Cilk & TBB/Cilk & PFunc/TBB \\
\hline
1  & 2.17 & 2.2178  & 4.431 & 0.5004 \\ 
\hline
2  & 1.15 & 2.1135 & 4.1924 & 0.5041 \\ 
\hline
4  & 0.55 & 2.2131 & 4.4183 & 0.5009 \\ 
\hline
8  & 0.28 & 2.2114 & 4.9839 & 0.4437 \\ 
\hline
16 & 0.15 & 2.4944 & 5.9370 & 0.4201 \\ 
\hline
\end{tabular}
\end{center}
\normalsize
\begin{itemize}
\item \textcolor{red}{2x faster than TBB!}
\item Only 2x slower than Cilk.
  \begin{itemize}
  \item \textcolor{red}{But provides more flexibility!}
    \begin{itemize}
    \item Fibonacci is the \textcolor{red}{worst} case behavior.
    \end{itemize}
  \item Library-based rather than a custom compiler.
  \end{itemize}
\end{itemize}
\tiny\textcolor{blue}{$\ast{}$4 socket, quad-core AMD 8356 with Linux 2.6.24.}\normalsize
\end{frame}

\begin{frame}
\frametitle{NAS Parallel Benchmarks.}
  \begin{itemize}
  \item Stands for \textit{N}ASA \textit{A}dvanced \textit{S}upercomputing.
  \item Help to evaluate performance of parallel tools and machines.
  \item Consits of 5 kernels and 3 psuedo applications.
    \begin{itemize}
    \item Taken mostly from Computational Fluid Dynamics (CFD).
    \end{itemize}
  \item Originally written in Fortran, but C versions are available.
  \item \textcolor{blue}{http://www.nas.nasa.gov/Resources/Software/npb.html}
  \item NPB OpenMP-C v2.3.
    \begin{itemize}
    \item Base code taken from Omni group's implementation.
    \end{itemize}
\end{itemize}
\end{frame}

\begin{frame}[fragile]
\frametitle{NAS Parallel Benchmarks.}
\tablefont
\begin{tabular}{|c|l|}
\hline
\textcolor{blue}{Benchmark} & Explanation  \\
\hline
\textcolor{blue}{Embarrassingly Parallel (EP)} & Gaussian random variates. \\
                                               & Marsaglia polar method. \\
\hline
\textcolor{blue}{Multigrid (MG)} & 3-dimensional discrete poisson equation. \\
\hline
\textcolor{blue}{Conjugate Gradient (CG)} & Iterative solver for linear systems. \\
                                          & Symmetric positive-definite matrices. \\
\hline
\textcolor{blue}{Integer sort (IS)} & Bucket sort.\\
\hline
\textcolor{blue}{LU Solver (LU)} & Lower-upped symmetric Gauss-Seidel. \\
                                 & System of nonlinear equations. \\
\hline
\textcolor{blue}{Pentadiagonal solver (SP)} & System of nonlinear equations. \\
\hline
\textcolor{blue}{Block tridiagonal solver (BT)} & System of nonlinear equations. \\
\hline
\end{tabular}
\normalsize
\end{frame}

\begin{frame}
\frametitle{The End}
  \begin{itemize}
  \item Modify data-parallel NPB OpenMP-C version to task parallel version.
  \item Compare against original NPB OpenMP-C version.
    \begin{itemize}
    \item For problem sizes in classes A, B and C.
    \end{itemize}
  \end{itemize}
\begin{center}
\textcolor{blue}{Questions?}
\end{center}
\end{frame}

\end{document}
